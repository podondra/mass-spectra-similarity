\documentclass[a4paper,10pt,twocolumn]{article}
\usepackage{lmodern}
\usepackage[czech]{babel}
\usepackage[T1]{fontenc}
\usepackage[utf8]{inputenc}
\usepackage{graphicx}
\usepackage{float}
\usepackage[top=0.5cm,bottom=2cm,left=1cm,right=1cm]{geometry}
\usepackage{amsmath}
\usepackage{amssymb}

\title{Similarity of Mass Spectra}
\date{\today}
\author{Ondřej Podsztavek, podszond@fit.cvut.cz}

\begin{document}

\maketitle

\section{Project Description}

The goal of this project is to create an application which searches
in database of protein sequences for peptide sequences similar to input
mass spectra.

\subsection{Input}

Applicaiton input is mass spectrum and database of protein
sequences.

\subsection{Ouput}

Output it set of peptide sequences similar to input mass
spectra sorted according to similarity.

\section{Solution Method}

Used methods and algorithms.

Similarity function \(\delta: \mathbb{U} \times \mathbb{U} \to \mathbb{R}\).

Cosine similarity computes the L2-normalized dot product of vectors. That is,
if \(x\) and \(y\) are vectors, their cosine similarity \(\text{SIM}_{\cos}\)
is defined as:

\[ \text{SIM}_{\cos}(x, y) = \frac{x^Ty}{\|x\|\|y\|} \]

Magnitude is not important.

\section{Implementation}

Description of used programming tools. Programming language, application
architecture, used libraries.

Python~3.6, scikit-learn~\cite{scikit-learn}, numpy, SciPy~\cite{scipy}, Flask,
pyteomics~\cite{Goloborodko2013}.

\section{Output Example}

\section{Experiments}

For kNN query the results are following. Write about the data used.

\begin{table}[h]
    \begin{center}
        \label{table:TODO}
        \begin{tabular}{l|rrr}
            & \multicolumn{3}{c}{\textbf{bin size}} \\
            \textbf{accuracy measure} & 0.1 & 0.5 & 1 \\
            \hline
            top 1 & 6.12 & 6.37 & 6.50 \\
            top 5 & 7.46 & 8.09 & 8.22 \\
            top 10 & 8.29 & 8.80 & 8.86 \\
        \end{tabular}
        \caption{
            Accuracy of similarity search in percents for different bin sizes.
            Top 10 accuracy measures if the searched peptide sequence is between
            the first 10 returned results. Accordingly are defined the top 5
            and top 1 accuracies.
        }
    \end{center}
\end{table}

\section{Discussion}

\section{Conclusion}

\bibliography{report}{}
\bibliographystyle{plain}

\end{document}
